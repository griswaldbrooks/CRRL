%%%%%%%%%%%%%%%%%%%%%%%%%%%%%%%%%%%%%%%%%%%%%%%%%%%%%%%%%%%%%%%%%%%%%%%%%%%%%%%%%%
%%% Introduction
%%%%%%%%%%%%%%%%%%%%%%%%%%%%%%%%%%%%%%%%%%%%%%%%%%%%%%%%%%%%%%%%%%%%%%%%%%%%%%%%%%
\chapter{Introduction}
	\label{ch::introduction}

		Then we stuck a Hokoyu laser on the Nao and used that for navigation. Why was this a good idea?

Took this stuff and did SLAM with it that we picked up from somewhere.

Used the camera data (and gyro) to detect openings using SFM.

Results from this work were published in \cite{our_paper1}.
And more results were published in newer papers.

This thesis is organized as follows: 
% Chapter 2 discusses the GODZILA path planning algorithm, Chapter 3 reviews applicable topics of the Nao platform, Chapter 4 presents simulation methodology and results, Chapter 5 presents results from 
% implementing the path planning algorithm on the Nao, and Chapter 6 states conclusions and closing remarks.
% 
% Chapter 3 Navigation 
% Chapter 4 Path Planning and Obstacle Avoidance
%  GODZILA, Multi Gait Planner, SFM, SLAM...
% Chapter 5 Projected Profile Gait
%  Gaiting problem, formulation, optimization structure
% Chapter 6 Experimental Results
% Chapter 7 Conclusion
Chapter 2 reviews the Nao Humanoid Platform with Hokoyu Scanning Laser Rangefinder augmentation.
The navigation system is broken into three parts, 
Chapter 3 gives a more in depth discussion of Particle Filter SLAM and the implemented algorithm, 
Chapter 4 talks about graph based path planning and the A* algorithm,
Chapter 5 discusses the GODZILA algorithm used for local navigation.
The crawl gait subsystem is discussed in three parts,
Chapter 6 discusses the Multi-Gait planner selecting the walking or crawling gait,
Chapter 7 discusses SFM for detecting openings that the Nao might crawl under,
Chapter 8 discusses the Projected Profile crawling gait used to perform the crawl.
Simulations and experimental results are shown in Chapters 9 and 10, while a discussion of the work is given in Chapter 11.