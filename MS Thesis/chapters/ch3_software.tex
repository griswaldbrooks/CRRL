
%%%%%%%%%%%%%%%%%%%%%%%%%%%%%%%%%%%%%%%%%%%%%%%%%%%%%%%%%%%%%%%%%%%%%%%%%%%%%%%%%%
%%% Software
%%%
%%% Section 1 : Software Architecture
%%% 
%%% Section 2 : Ground Station
%%%
%%% Section 3 : Simulator
%%%
%%%%%%%%%%%%%%%%%%%%%%%%%%%%%%%%%%%%%%%%%%%%%%%%%%%%%%%%%%%%%%%%%%%%%%%%%%%%%%%%%%
\chapter{Software}
	\label{ch::software}
	
	\section{System Software Architecture}
	
	BlueFoot is controlled using a multi-processor software architecture which incorporates several independent core programs. Each of these programs handles portions of system control in a distributed/cooperative fashion. Core programs are executed on physically separate operating units, allowing for low-level tasks, such as actuator command and feedback handling, battery monitoring, etc., to be decoupled from more computationally heavy tasks, such as high-level planning and navigation. With task-decoupling in mind, BlueFoot's software architecture is designed such that core programs could be readily offloaded to physically separate computing modules. Each of these control modules handles their own set of assigned tasks in independent control loops. Information is forwarded from each independent processor to update the overall BlueFoot software macro-system in an asynchronous fashion. System control tasks essential to BlueFoot's overall operation are divided into four main categories, which can be summarized as follows:
		\begin{itemize}
			\item{
			\emph{Low-Level Control} : 
				power monitoring/switching, 
				actuator command handling, 
				communications routing,
				sensor data acquisition,
				script parsing and evaluation
			}
			\item{
			\emph{Locomotion Control} : 
				gait planning, 
				gait adaptation, 
				trunk pose adaptation
			}
			\item{
			\emph{High-Level Control} : 
				perception,
				camera and LIDAR processing, 
				motion planning, 
				surface reconstruction, 
				navigation, 
				localization
			}
			\item{
			\emph{Human-Operator Control} : 
				joystick/keyboard commands,
				scripting commands,
				miscellaneous ground-station interaction
			}
		\end{itemize}
	Low-level and locomotion control tasks are handled, exclusively, by the \emph{Lower Brain} (LB), which designates a collection of software spanning over the RM48 and TM4C processors of the AutoPilot. High-level control tasks are handled by the Upper Brain (UB), which is a collection of software which runs on the ODROID-XU module. Lastly, a human operator can interface with the BlueFoot robot wirelessly from a personal computer running ground-station software. The ground station communicates with the BlueFoot robot over wireless communication lines routed to the TM4C processor and the ODROID-XU computer. The ground-station interfaces with the ODROID-XU over an SSH connection. This wireless tunnel connection is mainly used for configuring run-time settings and collecting data stored on-board the robot.

	Since this software architecture is distributed over several separate computational units, an integral part of this control architecture is an efficient, reconfigurable interprocessor communication protocol. Namely, interprocessor communication on board BlueFoot is performed using data packets transfered over serial lines. These data packets are formatted using an in-house designed binary-XML protocol, called EXI. This protocol facilitates a highly customizable packeting structure for asynchronous inter-module communication and utilizes robust packet-error checking routines. These sections will detail BlueFoot's interprocessor communication protocol, namely the composition of packets transferred between processor. 

	This section will also detail the specific software-level tasks handled by each of BlueFoot's processor; the speed at which each core software element is run (update frequency); and what data must be communicated between individual software nodes during operation. Additionally, this section will describe the ground-station software and corresponding user-interface utilized to control the BlueFoot quadruped and administer high-level commands.
	
	\subsection{System Task Allocation}

		\begin{figure}[h!]
			\centering
			\fbox{\includegraphics[width=\textwidth]{process_diagram.png}}
			\caption{BlueFoot's on-board processes/signals and their relationships.}
			\label{fig::process_diagram}
		\end{figure}

		Figure~\ref{fig::process_diagram} depicts how core software and associated control elements are related within the BlueFoot software macro-system. This section will detail the tasks carried out by each major software module implemented on the BlueFoot quadruped.

		\subsubsection{TM4C (Lower Brain)}

			As previously mentioned, the TM4C processor on-board the AutoPilot module is responsible for a majority of the systems \emph{Low-Level} tasks and can be viewed as a safety/communications routing co-processor. Within its main program loop, the TM4C polls the system's main battery voltage via an ADC interface; handles communication (packet decoding) routines between the ground-station and the RM48 system nodes; and handles command dispatching and feedback polling with the system's 16 servo actuators. The TM4C is directly interfaced with two dual-channel power switching IC's and is used to control the supply of power to each leg. Power is controlled by toggling general purpose I/O pins in software. The state of these pins is administered as part of a periodic packet command/update packet sent from the ground-station. Since the TM4C has control over BlueFoot's actuators (which consume most of the system's power) and battery monitoring capabilities, it runs a safety routine which is responsible for halting motor activity and cutting system power on low-battery or power-fault conditions as well as during unexpected breaks in communication with the ground-station.

			As previously mentioned, the TM4C handles communication routing among system processing modules as well as with the controllers on-board each smart-servo. Administering servo commands and collecting servo feedback are the TM4C's highest priority tasks. This process, which involves both commanding and requesting feedback from each servo, is relatively expensive  and limits the TM4C's main-loop frequency to roughly 50 Hz. Thus, it is particularly important that this task is offloaded to this processor. Other safety and communication-related tasks are much less expensive by comparison and allow the servo actuators to be updated as quickly as possible without encumbering other system control operations.


		\subsubsection{RM48 (Lower Brain)}

			The RM48 is responsible for several \emph{Low-Level} tasks, including IMU polling and handling communication with the TM4C and ODROID-XU. Each set of collected IMU data is passed along to an Extended Kalman Filter (EKF) routine, which is used to estimate that orientation of the BlueFoot's trunk.

			The RM48's primary function is to carry out motion control and gait-planning tasks. To achieve this, the RM48 handles a state machine which switches between planned motion execution and trajectory control; and a Central Pattern Generator (CPG) based gaiting controller, which will be discussed in more detail in Chapter~\ref{ch::gait_control}. Additional functions for body and posture (position and orientation) control including trunk control procedures and gait-stabilization routines are run in tandem with the aforementioned gait-control task.

			Motion and gait controls, which are performed in the robot task-space, are converted into joint-space reference angles, $q^{r}$, via an inverse kinematics (IK) routine. The IK routine is executed at all times when the legs are engaged for the purpose of issuing servo position commands. The RM48 also maintains BlueFoot's forward kinematic model, which is used to estimate the position of each foot relative to the trunk. This model relies on the EKF-generated trunk orientation estimate, $\hat{\theta}_{b}$, and joint position feedback, $q$. BlueFoot's inverse and forward kinematics models will be detailed in Chapter~\ref{ch::system_modeling}. The RM48 runs its full control loop at approximatively 100 Hz (twice the speed of the TM4C control loop) to facilitate higher integration stability when updating gait related controller dynamics, dynamic motion controls, and task-space reference trajectories.

			Lastly, the RM48 handles an on-board scripting engine (based on the MIT Squirrel scripting language), which interprets lexical commands. This scripting engine is capable of handling a large number of high-level commands and is complex enough to handle function and class definitions in real-time \cite{Squirrel_website}. The scripting engine is currently being used to evaluate BlueFoot's core user-command set, ranging from simple state toggling and parameter modification to the prescription of user-specified way-points for navigation, among other high-level command items. Scripting commands are passed from the ground station, via a text-input terminal, and routed through the TM4C to the RM48 where they are finally parsed and evaluated.

		\subsubsection{ODROID-XU (Upper Brain)}

			The ODROID-XU runs software upon the Debian (Linux) operating system distribution ``Jessie." The use of an Linux operating system extends itself to a number of conveniences, such as the ability to run several tasks in parallel threads. In-built USB-serial drivers are used to acquire data from USB-interfaced vision sensors. Namely, the ORDOID handles the acquisition of camera data and controlling camera frame-rate; as well as the acquisition of LIDAR data. The ORI/OID uses these sensor inputs to perform several navigation-related tasks. These tasks will be described in more detail in Chapter~\ref{ch::navigation}.

			The ODROID-XU utilizes 2D-LIDAR scans (frames) and trunk-pose estimates to form organized 3D point clouds. These point-clouds are further processed to reconstruct 3D terrain surfaces and height-maps, which are then used for step-planning. LIDAR and camera data is utilized in a potential fields-based navigation routine. Image processing and image feature detection runs as a separate process on the ODROID. In particular, the open-source libraries OpenCV (Open Computer Vision Library), OpenPCL (Open Point-Cloud Library), and Boost are used heavily in the software developed to carry out the aforementioned tasks \cite{opencv_library,openpcl_library,boost_website}. Collectively, software written for this platform was generated using a mixture of C++ and Python.

			As previously mentioned, the ODROID-XU can handle a limited set of user-commands on its own, which are administered directly to the ODROID-XU from an SSH terminal on the ground-station computer. These commands include core-program start-ups and run-time configurators. Essentially, the ODROID's software core is designed as a completely independent software module which replaces the roll of a human operator, as it handles the bulk of the systems high-level planning and navigation tasks. Moreover, if the ODROID-XU is removed from the BlueFoot system, the system can still be operated via remote-control heading commands provided from human operator using BlueFoot's ground-station joystick control inputs.
		
		\subsection{Inter-Processor Communication}

			\begin{figure}[h!]
				\centering
				\fbox{\includegraphics[width=\textwidth]{comm_flow.png}}
				\caption{Communication flow among processors on the BlueFoot platform.}
				\label{fig::comm_flow}
			\end{figure}

			This section will detail the contents of the data packets transfered among processors, which is summarized in Figure~\ref{fig::comm_flow}. System directives, generated by a human operator who interacts with the robot via a graphical user interface and joystick controller, originate from a ground-station computer. Update packets which are sent from the ground-station to the BlueFoot robot (TM4C) are composed as shown in Table~\ref{tab::gs_to_tm4c_packet}:
			\begin{table}[h!]
				\centering
				\begin{tabularx}{\textwidth}{|C{0.2}|C{0.2}|C{0.2}|C{0.2}|C{0.2}|} 	
					\hline
					\emph{32-bits} 	& \emph{8-bits} 		& \emph{8-bits} 	&\emph{16-bits} 	& \emph{16-bits} 	\\\hline
					HEADER 		& Master-Tog.		& Power-Tog.	& Unused		& Network Info 	\\\hline
					\emph{Variable} 	& 		 		& 			&			& 			\\\hline
					Scripts 		& 				& 			& 			&			\\\hline
				\end{tabularx} 
				\caption{Structure of the packets sent from Ground-Station to TM4C.}
				\label{tab::gs_to_tm4c_packet}
			\end{table}
			
			Every packet issued has a 4-byte (32-bit) header, which is added as part of the EXI protocol. As part of the internal system protocol, every packet contains a ``Master Status Vector", which is comprised of a fixed length, 7-byte sequence of essential system information and control states. The ``Master Toggles" section (first 8-bits after header) enumerates major systems states, including \emph{On-line, Standby, Off-line} and \emph{Suspended} system state designations. The next 8-bits are used to toggle on-board power, namely the power supplied to each leg. The remaining 32-bits are used for specifying battery voltage (8-bits), power-fault states (8-bits); generic binary feedback toggles (8-bits) used to store foot contact information; and system networking information (16-bits). The last section of this packet contains scripting commands, which vary in length. Forward velocity and turning rate commands (gathered form a joy-stick control input) are administered in the form of scripted commands.

			Packets sent from the TM4C to the Ground-station contain status items which are maintained on-board the robot, and appear as shown in Table~\ref{tab::tm4c_to_gs_packet}:
			\begin{table}[h!]
				\centering
				\begin{tabularx}{\textwidth}{|C{0.2}|C{0.2}|C{0.2}|C{0.2}|C{0.2}|} 	
					\hline
					\emph{32-bits} 	& \emph{8-bits} 		& \emph{8-bits} 	& \emph{8-bits} 	& \emph{16-bits} 	\\\hline
					HEADER 		& Master-Tog.		& Unused		& Foot-Contacts	& Network Info 	\\\hline\hline
					\emph{2048-bits} 	& 				&			&  			& 		 	\\\hline
					Joint Pos. FB		& 				& 			& 			& 			\\\hline
				\end{tabularx} 
				\caption{Structure of the packets sent from the TM4C to the Ground-Station.}
				\label{tab::tm4c_to_gs_packet}
			\end{table}
			The \emph{Joint Pos. FB} (joint position feedback) element is composed of 32-bytes corresponding to the joint position feedback from each actuator. To avoid redundancy, the structure of the packets communicated from the TM4C to the RM48 and vice-versa will not be depicted explicitly. These packets contain a 7-byte master status vector with only the \emph{Master-Toggle} and \emph{Network Info} fields populated. The TM4C sends the same joint feedback information to the RM48 as it does the Ground Station. For packets sent from the TM4C to the RM48, this field is replaced by a 32-byte sequence of joint-position commands.

			Packets sent from the RM48 to the ODROID-XU contain additional fields which contain dynamical-state information used in planning on the ODROID. State information is sent in the form of vectors with single precision floating-point (32-bit) elements. Such information includes trunk orientation estimates, angular rate, and global position (generated from open-loop command integration); and four foot-position estimates, each of which are represented as a separate 3-element vectors. These packets have a structure which is depicted in Table~\ref{tab::rm48_to_odroid}.
%
				\begin{table}[h!]
					\centering
					\begin{tabularx}{\textwidth}{|C{0.2}|C{0.2}|C{0.2}|C{0.2}|C{0.2}|} 	
						\hline
						\emph{32-bits} 	& \emph{8-bits} 		& \emph{8-bits} 	& \emph{8-bits} 	& \emph{16-bits} 	\\\hline
						HEADER 		& Master-Tog.		& Unused		& Foot-Contacts	& Network Info 	\\\hline\hline
						\emph{96-bits} 	& \emph{96-bits}		& \emph{328-bits}	&\emph{328-bits}  	& 		 	\\\hline
						Orientation		& Angular Rate		& Trunk Pos.		& Foot Positions	& 			\\\hline
					\end{tabularx} 
					\caption{Structure of the packets sent from the RM48 to the ODROID.}
					\label{tab::rm48_to_odroid}
				\end{table}
			Packets sent form the ODROID-XU to the RM48 contain command items, such as forward velocity, turning rate and trunk-pose commands, as well as foothold-references and corrected global trunk position estimates. These packets are constructed as shown in Table~\ref{tab::odroid_to_rm48}:
	%
				\begin{table}[h!]
					\centering
					\begin{tabularx}{\textwidth}{|C{0.2}|C{0.2}|C{0.2}|C{0.2}|C{0.2}|} 	
						\hline
						\emph{32-bits} 	& \emph{8-bits} 		& \emph{8-bits} 	& \emph{8-bits} 	& \emph{16-bits} 	\\\hline
						HEADER 		& Master-Tog.		& Unused		& Unused		& Network Info 	\\\hline\hline
						\emph{32-bits} 	& \emph{32-bits}		& \emph{96-bits}	&\emph{328-bits}  	& \emph{96-bits} 	\\\hline
						Velocity		& Turning Rate		& Trunk Pose		& Footholds.	& Trunk Pos.		\\\hline
					\end{tabularx} 
					\caption{Structure of the packets sent from the ODROID-XU to the RM48.}
					\label{tab::odroid_to_rm48}
				\end{table}

	\section{Ground Station}
		
		Ground-station software used for controlling the BlueFoot platform is written in C++ using an open-source graphical user interface design library called \emph{wxWidgets} \cite{WX_Website}. \emph{wxWidgets} is used, primarily, for the ground-station's front-end design. Namely, the ground-station code is designed such that its user-interface design is reconfigurable. This is made possible by an XML-based design configurator, which can be used to change the look and location of buttons and panels which make up the user-interface, without having to modify the ground-station back-end software. 

		The back-end portion of the ground-station software handles interrupts generated by user inputs, such as button presses and text input. \emph{wxWidgets} make the process of associating functional callbacks to user input events relatively easy. Additionally, \emph{wxWidgets} provides an interface for collecting joystick inputs, which are interpreted and sent to the BlueFoot as scripting commands. \emph{Boost}, a C++ utility library, is utilized to handle socket and serial-port I/O controls. This library is particularly important handling serial-I/O ports on the ground station computer, which are used in communicating with the robot over a wireless radio connection.

		The ground station is composed of several main sections of user-interface which provide interfaces for administering system master-state, operating-state and power toggling; providing system commands; and scripting. Ground station commands are sent to the robot in order to modify operating states and administer manual navigation commands. Updates from the ground-station are sent to the platform at a rate of 25 Hz using an XBEE radio module. BlueFoot replies to each ground-station update with a packet of internal configurations, as shown in Table~\ref{tab::tm4c_to_gs_packet}, which is then used to check that system updates and commands have been received and that the system is live.

		BlueFoot sends the following particles of system information back to the ground-station at a user-specified rate: 
			\begin{itemize}
				\item battery voltage levels 
				\item power fault conditions 
				\item foot contact feedback
				\item joint position feedback. 
			\end{itemize}
		This data is used to update several key portions of the user-interface. Battery levels are used to update a battery meter, which indicates the current system voltage level. Additionally, a dynamic text box displays the system's power-fault state to the user. Joint position and foot-contact information is used to update a visualization of the robot in real-time. Joint positions, foot-contact states and battery levels are time-stamped and automatically logged during each ground-station session. 

